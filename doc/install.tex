%%%%%%%%%%%%%%%%%%%%%%%%%%%%%%%%%%%%%%%%%%%%%%%%%%%%%%%%%%%%%%%%%%%%%%%%%
%%
%W  install.tex            GAP documentation            Christian Sievers
%%
%H  $Id$
%%
%Y  2003
%%

%%%%%%%%%%%%%%%%%%%%%%%%%%%%%%%%%%%%%%%%%%%%%%%%%%%%%%%%%%%%%%%%%%%%%%%%%
\Chapter{Installing and loading the FGA package}

\atindex{Installing and loading the FGA package}{@installing %
                        and loading the {\FGA} package}
%%%%%%%%%%%%%%%%%%%%%%%%%%%%%%%%%%%%%%%%%%%%%%%%%%%%%%%%%%%%%%%%%%%%%%%%%
\Section{Installing the FGA package}\null

\atindex{Installing the FGA package}{@installing the {\FGA} package}
The installation of the {\FGA} package follows standard {\GAP} rules.
So the standard method is to unzoo the package into the `pkg'
directory  of your {\GAP} distribution.  This will create an `fga'
subdirectory. 

For other non-standard options please see Chapter~"ref:Installing a
GAP Package" in the {\GAP} Reference Manual.

%To create the documentation, go into the `doc' directory and type
%`make_doc'.

%%%%%%%%%%%%%%%%%%%%%%%%%%%%%%%%%%%%%%%%%%%%%%%%%%%%%%%%%%%%%%%%%%%%%%%%%
\Section{Loading the FGA package}\null

\atindex{Loading the FGA package}{@loading the {\FGA} package}
To use the {\FGA} Package you have to request it explicitly. This  is
done by calling `LoadPackage' like this:

\beginexample
gap> LoadPackage("fga");
-----------------------------------------------------------------------------
Loading  FGA 1.0 (Free Group Algorithms)
by Christian Sievers (c.sievers@tu-bs.de).
-----------------------------------------------------------------------------
true
\endexample

The `LoadPackage' command is described in Section~"ref:LoadPackage"
in the {\GAP} Reference Manual.

%%%%%%%%%%%%%%%%%%%%%%%%%%%%%%%%%%%%%%%%%%%%%%%%%%%%%%%%%%%%%%%%%%%%%%%%%
%%
%E
