%%%%%%%%%%%%%%%%%%%%%%%%%%%%%%%%%%%%%%%%%%%%%%%%%%%%%%%%%%%%%%%%%%%%%%%%%
%%
%W  intro.tex          FGA documentation                Christian Sievers
%%
%H  $Id$
%%
%Y  2003
%%

%%%%%%%%%%%%%%%%%%%%%%%%%%%%%%%%%%%%%%%%%%%%%%%%%%%%%%%%%%%%%%%%%%%%%%%%%
\Chapter{Introduction}

\atindex{FGA}{@FGA}

%%%%%%%%%%%%%%%%%%%%%%%%%%%%%%%%%%%%%%%%%%%%%%%%%%%%%%%%%%%%%%%%%%%%%%%%%
\Section{Overview}

This manual describes the {\FGA} (*Free Group Algorithms*) package,
a {\GAP} package for computations with finitely generated subgroups of
free groups.

This package allows you to test membership and conjugacy, and to compute
free generators, the rank, the normalizer, centralizer, and index,
where the groups involved are finitely generated subgroups of free groups.
%
In addition, it provides a finite presentation for the
automorphism group of a finitely generated free group following
\cite{Neumann33}.

See Chapter "Functionality of the FGA Package" for details.

Chapter "Installing and Loading the FGA Package" explains
how to install and load the {\FGA} package.

%%%%%%%%%%%%%%%%%%%%%%%%%%%%%%%%%%%%%%%%%%%%%%%%%%%%%%%%%%%%%%%%%%%%%%%%%
\Section{Implementation and background}

The methods which are used work mainly with inverse finite automata,
a variant of a concept known from theoretical computer science.

Most of these techniques are described in \cite{Sims94}.

In \cite{BirgetEtAl00}, the connection between finitely generated
subgroups of free groups and inverse finite automata is used to transfer
results about the space complexity of problems concerning inverse finite
automata to analogous results about finitely generated subgroups of free
groups.

Some word oriented algorithms in the {\FGA} package use basic facts about
free groups that can for example be found in \cite{LyndonSchupp77}.

The theoretical background for this implementation is explained
in \cite{Sievers03}.

%%%%%%%%%%%%%%%%%%%%%%%%%%%%%%%%%%%%%%%%%%%%%%%%%%%%%%%%%%%%%%%%%%%%%%%%%
%%
%E
