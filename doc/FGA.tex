%%%%%%%%%%%%%%%%%%%%%%%%%%%%%%%%%%%%%%%%%%%%%%%%%%%%%%%%%%%%%%%%%%%%%%%%%
%%
%W  FGA.tex            FGA documentation                Christian Sievers
%%
%H  $Id$
%%
%Y  2003 - 2012
%%

%%%%%%%%%%%%%%%%%%%%%%%%%%%%%%%%%%%%%%%%%%%%%%%%%%%%%%%%%%%%%%%%%%%%%%%%%
\Chapter{Functionality of the FGA package}

\atindex{Functionality of the FGA package}{@Functionality %
                                           of the {\FGA} package|indexit}

This chapter describes methods available from the {\FGA} package.

In the following, let <f> be a free group created by `FreeGroup(<n>)',
and let <u>, <u1> and <u2> be finitely generated subgroups of <f>
created by `Group' or `Subgroup', or computed from some other subgroup
of <f>.  Let <elm> be an element of <f>.

For example:

\beginexample
gap> f := FreeGroup( 2 );                                             
<free group on the generators [ f1, f2 ]>
gap> u := Group( f.1^2, f.2^2, f.1*f.2 );
Group([ f1^2, f2^2 ])
gap> u1 := Subgroup( u, [f.1^2, f.1^4*f.2^6] );
Group([ f1^2, f1^4*f2^6 ])
gap> elm := f.1;
f1
gap> u2 := Normalizer( u, elm );
Group([ f1^2 ])
\endexample

%%%%%%%%%%%%%%%%%%%%%%%%%%%%%%%%%%%%%%%%%%%%%%%%%%%%%%%%%%%%%%%%%%%%%%%%%
\Section{New operations for free groups}

These new operations are available for finitely generated subgroups of
free groups:

\>FreeGeneratorsOfGroup( <u> ) A

returns a list of free generators of the finitely generated subgroup
<u> of a free group.

The elements in this list form an N-reduced set.  In addition to
being a free (and thus minimal) generating set for <u>, this means
that whenever <v1>, <v2> and <v3> are elements or inverses of elements
of this list, then

\beginlist%unordered
  \item{--}
    $<v1><v2>\neq1$ implies $|<v1><v2>|\geq\max(|<v1>|, |<v2>|)$, and
  \item{--}
    $<v1><v2>\neq1$ and $<v2><v3>\neq1$ implies
    $|<v1><v2><v3>| > |<v1>| - |<v2>| + |<v3>|$
\endlist

hold, where $|.|$ denotes the word length.

\>RankOfFreeGroup( <u> ) A
\>Rank( <u> ) O

returns the rank of the finitely generated subgroup <u> of a free
group.

\>CyclicallyReducedWord( <elm> ) O

returns the cyclically reduced form of <elm>.

%%%%%%%%%%%%%%%%%%%%%%%%%%%%%%%%%%%%%%%%%%%%%%%%%%%%%%%%%%%%%%%%%%%%%%%%%
\Section{Method installations}

This section lists operations that are already known to {\GAP}.
{\FGA} installs new methods for them so that they can also be used
with free groups.
In cases where {\FGA} installs methods that are usually only used
internally, user functions are shown instead.


\>Normalizer( <u1>, <u2> ) O
\>Normalizer( <u>, <elm> ) O

The first variant returns the normalizer of the finitely generated
subgroup <u2> in <u1>.

The second variant returns the normalizer of $\langle <elm> \rangle$
in the finitely generated subgroup <u> (see "ref:Normalizer" in the
Reference Manual).

\>RepresentativeAction( <u>, <d>, <e> ) O
\>IsConjugate( <u>, <d>, <e> ) O

`RepresentativeAction' returns an element $ <r> \in <u> $,
where <u> is a finitely generated subgroup of a free group, such
that $<d>^{<r>}=<e>$, or fail, if no such <r> exists.  <d> and <e> may
be elements or subgroups of <u>.

`IsConjugate' returns a boolean indicating whether such an element <r>
exists.

\>Centralizer( <u>, <u2> ) O
\>Centralizer( <u>, <elm> ) O

returns the centralizer of <u2> or <elm> in the finitely generated
subgroup <u> of a free group.

\>Index( <u1>, <u2> ) O
\>IndexNC( <u1>, <u2> ) O

return the index of <u2> in <u1>, where <u1> and <u2> are finitely
generated subgroups of a free group.  The first variant returns
fail if <u2> is not a subgroup of <u1>, the second may return
anything in this case.

\>Intersection( <u1>, <u2> \dots ) F

returns the intersection of <u1> and <u2>, where <u1> and <u2> are
finitely generated subgroups of a free group.

\>`<elm> in <u>'{in} O

tests whether <elm> is contained in the finitely generated subgroup
<u> of a free group.

\>IsSubgroup( <u1>, <u2> ) F

tests whether <u2> is a subgroup of <u1>, where <u1> and <u2> are finitely
generated subgroups of a free group.

\>`<u1> = <u2>'{equality} O

test whether the two finitely generated subgroups <u1> and <u2> of a
free group are equal.

\>MinimalGeneratingSet( <u> ) A
\>SmallGeneratingSet( <u> ) A
\>GeneratorsOfGroup( <u> ) A

return generating sets for the finitely generated subgroup <u> of a
free group.  `MinimalGeneratingSet' and `SmallGeneratingSet' return
the same free generators as `FreeGeneratorsOfGroup', which are in
fact a minimal generating set.  `GeneratorsOfGroup' also returns these
generators, if no other generators were stored at creation time.

%%%%%%%%%%%%%%%%%%%%%%%%%%%%%%%%%%%%%%%%%%%%%%%%%%%%%%%%%%%%%%%%%%%%%%%%%
\Section{Constructive membership test}

It is not only possible to test whether an element is in a finitely
generated subgroup of free group, this can also be done
constructively.  The idiomatic way to do so is by using a
homomorphism.

Here is an example that computes how to write `f.1^2' in the
generators `a=f1^2*f2^2' and `b=f.1^2*f.2', checks the result, and
then tries to write `f.1' in the same generators:

\beginexample
gap> f := FreeGroup( 2 );
<free group on the generators [ f1, f2 ]>
gap> ua := f.1^2*f.2^2;; ub := f.1^2*f.2;;
gap> u := Group( ua, ub );;
gap> g := FreeGroup( "a", "b" );;
gap> hom := GroupHomomorphismByImages( g, u,
              GeneratorsOfGroup(g),
              GeneratorsOfGroup(u) );
[ a, b ] -> [ f1^2*f2^2, f1^2*f2 ]
gap> # how can f.1^2 be expressed?
gap> PreImagesRepresentative( hom, f.1^2 );
b*a^-1*b
gap> last ^ hom; # check this
f1^2
gap> ub * ua^-1 * ub; # another check
f1^2
gap> PreImagesRepresentative( hom, f.1 ); # try f.1
fail
gap> f.1 in u;
false
\endexample

There are also lower level operations to get the same results.
In fact, they are faster when used repeatedly with the same group.

\>AsWordLetterRepInGenerators( <elm>, <u> ) O
\>AsWordLetterRepInFreeGenerators( <elm>, <u> ) O

return a letter representation
(see Section~"ref:Representations for Associative Words" in the {\GAP}
Reference Manual)
of the given <elm> relative to the generators the group was created
with or the free generators as returned by `FreeGeneratorsOfGroup'.

Continuing the above example:

\beginexample
gap> AsWordLetterRepInGenerators( f.1^2, u );    
[ 2, -1, 2 ]
gap> AsWordLetterRepInFreeGenerators( f.1^2, u );
[ 2 ]
\endexample

This means: to get `f.1^2', multiply the second of the given generators
with the inverse of the first and again with the second; or just take
the second free generator.

%%%%%%%%%%%%%%%%%%%%%%%%%%%%%%%%%%%%%%%%%%%%%%%%%%%%%%%%%%%%%%%%%%%%%%%%%
\Section{Automorphism groups of free groups}

The {\FGA} package knows presentations of the automorphism groups of free
groups. It also allows to express an automorphism as word in the
generators of these presentations.
This sections repeats the {\GAP} standard methods to do so and shows
functions to obtain the generating automorphisms.

\>AutomorphismGroup( <u> ) A

returns the automorphism group of the finitely generated subgroup <u>
of a free group.

Only a few methods will work with this group. But there is a way to
obtain an isomorphic finitely presented group:

\>IsomorphismFpGroup( <group> ) A

returns an isomorphism of <group> to a finitely presented group.  
For automorphism groups of free groups, the {\FGA} package implements
the presentations of \cite{Neumann33}.
The finitely presented group itself can then be obtained with the
command `Range'.

Here is an example:

\beginexample
gap> f := FreeGroup( 2 );;
gap> a := AutomorphismGroup( f );;
gap> iso := IsomorphismFpGroup( a );;
gap> Range( iso );
<fp group on the generators [ O, P, U ]>
\endexample

To express an automorphism as word in the generators of the
presentation, just apply the isomorphism obtained from
`IsomorphismFpGroup'.

\beginexample
gap> aut := GroupHomomorphismByImages( f, f,
               GeneratorsOfGroup( f ), [ f.1^f.2, f.1*f.2 ] );
[ f1, f2 ] -> [ f2^-1*f1*f2, f1*f2 ]
gap> ImageElm( iso, aut );
O^2*U*O*P^-1*U
\endexample

It is also possible to use `aut^iso' or `Image( iso, aut )'.
Using `Image' will perform additional checks on the arguments.

The {\FGA} package provides a simpler way to create endomorphisms:

\>FreeGroupEndomorphismByImages( <g>, <imgs> ) F

returns the endomorphism that maps the free generators of the finitely
generated subgroup <g> of a free group to the elements listed in <imgs>.
You may then apply `IsBijective' to check whether it is an
automorphism.

The follwowing functions return automorphisms that correspond to the
generators in the presentation:

\>FreeGroupAutomorphismsGeneratorO( <group> ) F
\>FreeGroupAutomorphismsGeneratorP( <group> ) F
\>FreeGroupAutomorphismsGeneratorU( <group> ) F
\>FreeGroupAutomorphismsGeneratorS( <group> ) F
\>FreeGroupAutomorphismsGeneratorT( <group> ) F
\>FreeGroupAutomorphismsGeneratorQ( <group> ) F
\>FreeGroupAutomorphismsGeneratorR( <group> ) F

return the automorphism which maps the free generators 
[$ x_1, x_2, \dots, x_n $] of <group> to
\beginlist
  \item{O:} [$ x_1^{-1}, x_2, \dots, x_n $]                      ($n\ge1$)
  \item{P:} [$ x_2, x_1, x_3, \dots, x_n $]                      ($n\ge2$)
  \item{U:} [$ x_1x_2, x_2, x_3, \dots, x_n $]                   ($n\ge2$)
  \item{S:} [$ x_2^{-1}, x_3^{-1}, \dots, x_n^{-1}, x_1^{-1} $]  ($n\ge1$)
  \item{T:} [$ x_2, x_1^{-1}, x_3, \dots, x_n $]                 ($n\ge2$)
  \item{Q:} [$ x_2, x_3, \dots, x_n, x_1 $]                      ($n\ge2$)
  \item{R:} [$ x_2^{-1}, x_1, x_3, x_4, \dots,
             x_{n-2}, x_nx_{n-1}^{-1}, x_{n-1}^{-1} $]           ($n\ge4$)
\endlist


%%%%%%%%%%%%%%%%%%%%%%%%%%%%%%%%%%%%%%%%%%%%%%%%%%%%%%%%%%%%%%%%%%%%%%%%%
%%
%E

