%%%%%%%%%%%%%%%%%%%%%%%%%%%%%%%%%%%%%%%%%%%%%%%%%%%%%%%%%%%%%%%%%%%%%%%%%
%%
%W  FGA.tex            FGA documentation                Christian Sievers
%%
%H  $Id$
%%
%Y  2003
%%

%%%%%%%%%%%%%%%%%%%%%%%%%%%%%%%%%%%%%%%%%%%%%%%%%%%%%%%%%%%%%%%%%%%%%%%%%
\Chapter{Functionality of the FGA package}

\atindex{Functionality of the FGA package}{@functionality%
                                            of the {\FGA} package}

This chapter describes methods available from the {\FGA} package.

In the following, let <f> be a free group created by `FreeGroup(<n>)',
and let <u>, <u1> and <u2> be finitely generated subgroups of <f>
created by `Group' or `Subgroup', or computed from some other subgroup
of <f>.  Let <elm> be an element of <f>.

For example:

\beginexample
gap> f := FreeGroup( 2 );                                             
<free group on the generators [ f1, f2 ]>
gap> u := Group( f.1^2, f.2^2, f.1*f.2 );
Group([ f1^2, f2^2 ])
gap> u1 := Subgroup( u, [f.1^2, f.1^4*f.2^6] );
Group([ f1^2, f1^4*f2^6 ])
gap> elm := f.1;
f1
gap> u2 := Normalizer( u, elm );
Group([ f1^2 ])
\endexample

%%%%%%%%%%%%%%%%%%%%%%%%%%%%%%%%%%%%%%%%%%%%%%%%%%%%%%%%%%%%%%%%%%%%%%%%%
\Section{New operations for free groups}

These new operations are available for finitely generated subgroups of
free groups:

\>FreeGeneratorsOfGroup( <u> ) A

returns a list of free generators of the finitely generated free group
<u>.

The elements in this list will form an N-reduced set.  In addition to
being a free (and thus minimal) generating set for <u>, this means
that whenever <v1>, <v2> and <v3> are elements or inverses of elements
of this list, then

\beginlist%unordered
  \item{--}
    $<v1><v2> \neq 1$ implies $|<v1><v2>| \geq \max(|<v1>|, |<v2>|)$, and
  \item{--}
    $<v1><v2> \neq 1$ and $<v2><v3> \neq 1$ implies
    $|<v1><v2><v3>| > |<v1>| - |<v2>| + |<v3>|$
\endlist

hold, where $|.|$ denotes the word length.

\>RankOfFreeGroup( <u> ) A
\>Rank( <u> ) O

returns the rank of the finitely generated free group <u>.

%%%%%%%%%%%%%%%%%%%%%%%%%%%%%%%%%%%%%%%%%%%%%%%%%%%%%%%%%%%%%%%%%%%%%%%%%
\Section{Method installations}

This section list operations that are already known to {\GAP}.
{\FGA} installs new methods for them so that they can also be used
with free groups.

\>Normalizer( <u1>, <u2> ) O
\>Normalizer( <u>, <elm> ) O

The first variant returns the normalizer of the finitely generated
subgroup <u2> in <u1>.

The second variant returns the normalizer of $\langle <elm> \rangle$
in the finitely generated subgroup <u> (see "ref:Normalizer" in the
Reference Manual).

\>RepresentativeAction( <u>, <d>, <e> ) O
\>IsConjugate( <u>, <d>, <e> ) O

`RepresentativeAction' returns an element $ <r> \in <u> $,
where <u> is a finitely generated subgroup of a free group, such
that $<d>^{<r>}=<e>$, or fail, if no such <r> exists.  <d> and <e> may
be elements or subgroups of <u>.

`IsConjugate' returns a boolean indicating whether such an element <r>
exists.

\>Centralizer( <u>, <u2> ) O
\>Centralizer( <u>, <elm> ) O

returns the centralizer of <u2> or <elm> in the finitely generated
subgroup <u> of a free group.

\>Index( <u1>, <u2> ) O
\>IndexNC( <u1>, <u2> ) O

return the index of <u2> in <u1>, where <u1> and <u2> are finitely
generated subgroups of a free group.  The first variant returns
fail if <u2> is not a subgroup of <u1>, the second may return
anything in this case.

%%%%%%%%%%%%%%%%%%%%%%%%%%%%%%%%%%%%%%%%%%%%%%%%%%%%%%%%%%%%%%%%%%%%%%%%%
\Section{Automorphism groups of free groups}

The {\FGA} package knows how to compute automorphism groups of free
groups. This sections repeats the {\GAP} standard methods to obtain them.

\>AutomorphismGroup( <u> ) A

returns the automorphism group of the finitely generated subgroup <u>
of a free group.

Only a few methods will work with this group. But there is way to
obtain an isomorphic finitely presented group:

\>IsomorphismFpGroup( <group> ) A

returns an isomorphism of <group> to a finitely presented group.  
For automorphism groups of free groups, the {\FGA} package implements
the presentation of \cite{Neumann33}.
That finitely presented group itself can then be obtained with the
command `Range'.

Here is an example:

\beginexample
gap> f := FreeGroup( 2 );
<free group on the generators [ f1, f2 ]>
gap> a := AutomorphismGroup( f );
<group of size infinity with 3 generators>
gap> iso := IsomorphismFpGroup( a );
[ [ f1, f2 ] -> [ f1^-1, f2 ], [ f1, f2 ] -> [ f2, f1 ], 
  [ f1, f2 ] -> [ f1*f2, f2 ] ] -> [ O, P, U ]
gap> Range( iso );
<fp group on the generators [ O, P, U ]>
\endexample


%%%%%%%%%%%%%%%%%%%%%%%%%%%%%%%%%%%%%%%%%%%%%%%%%%%%%%%%%%%%%%%%%%%%%%%%%
%%
%E

