%%%%%%%%%%%%%%%%%%%%%%%%%%%%%%%%%%%%%%%%%%%%%%%%%%%%%%%%%%%%%%%%%%%%%%%%%
%%
%W  FGA.tex            FGA documentation                Christian Sievers
%%
%H  $Id$
%%
%Y  2003
%%

%%%%%%%%%%%%%%%%%%%%%%%%%%%%%%%%%%%%%%%%%%%%%%%%%%%%%%%%%%%%%%%%%%%%%%%%%
\Chapter{The FGA Package}

\atindex{FGA package}{@FGA package}
This chapter  describes  the  {\FGA}  package.

In the following, let <f> be a free group created by `FreeGroup(<n>)',
and let <u>, <u1> and <u2> be finitely generated subgroups of <f>
created by `Group' or `Subgroup', or computed from some other subgroup
<f>.  Let <elm> be an element of <f>.

For example:

\beginexample
gap> f := FreeGroup( 2 );                                             
<free group on the generators [ f1, f2 ]>
gap> u := Group( f.1^2, f.2^2, f.1*f.2 );
Group([ f1^2, f2^2 ])
gap> u1 := Subgroup( u, [f.1^2, f.1^4*f.2^6] );
Group([ f1^2, f1^4*f2^6 ])
gap> elm := f.1;
f1
gap> u2 := Normalizer( u, elm);
Group([ f1^2 ])
\endexample

%%%%%%%%%%%%%%%%%%%%%%%%%%%%%%%%%%%%%%%%%%%%%%%%%%%%%%%%%%%%%%%%%%%%%%%%%
\Section{New functions}

These new functions are available for finitely generated subgroups of
free groups:

\>FreeGeneratorsOfGroup( <u> ) A

returns a list of free generators of the free group <u>.
This will be an N-reduced set...

\>RankOfFreeGroup( <u> ) A
\>Rank( <u> ) O

returns the rank of the free group <u>.

\Section{Method installations}

This section list methods that are already known to {\GAP}, for which
{\FGA} installs methods so that they can also be used with free groups.

\>Normalizer( <f>, <u> ) O
\>Normalizer( <u>, <elm> ) O

The first variant returns the normalizer of <u> in <f>, (this is planned
to be generalized to work with other groups than the whole family as
first argument later)

The second veriant returns the normalizer of <elm> in <u>.

\>RepresentativeAction( <u>, <d>, <e> ) O
\>IsConjugate( <u>, <d>, <e> ) O

`RepresentativeAction' returns an element $ <r> \in <group> $ such
that $<d>^<r>=<e>$, or fail, if no such <r> exists.  <d> and <e> may
be elements or subgroups of <u>, in the latter case <u> has to be the
whole family (this is plannend to be generalized later).

`IsConjugate' returns a boolean indicating whether such an element <r>
exists.

\>Centralizer( <u>, <u2> ) O
\>Centralizer( <u>, <elm> ) O

returns the centralizer of <u2> or <elm> in the free group <u>.

\>Index( <u1>, <u2> ) O
\>IndexNC( <u1>, <u2> ) O

return the index of <u2> in <u1>.  The first variant returns
fail if <u2> is not a subgroup of <u1>, the second may return
anything in this case.

\Section{Automorphism Groups of free groups}

The {\FGA} package knows how to compute automorphism groups of free
groups. This sections repeats the {\GAP} standard methods to obtain them.

\>AutomorphismGroup( <u> ) A

returns the automorphism group of the free group <u>.

\>IsomorphismFpGroup( <group> ) A

returns an isomorphism of <group> to a finitely presented <group>.  That
finitely presented group itself can then be obtained with the command
`Range'.

Here is an example:

\beginexample
gap> f := FreeGroup( 2 );
<free group on the generators [ f1, f2 ]>
gap> a := AutomorphismGroup( f );
<group of size infinity with 3 generators>
gap> iso := IsomorphismFpGroup( a );
[ [ f1, f2 ] -> [ f1^-1, f2 ], [ f1, f2 ] -> [ f2, f1 ], 
  [ f1, f2 ] -> [ f1*f2, f2 ] ] -> [ O, P, U ]
gap> Range( iso );
<fp group on the generators [ O, P, U ]>
\endexample


%%%%%%%%%%%%%%%%%%%%%%%%%%%%%%%%%%%%%%%%%%%%%%%%%%%%%%%%%%%%%%%%%%%%%%%%%
%%
%E

